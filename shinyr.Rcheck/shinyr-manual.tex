\nonstopmode{}
\documentclass[a4paper]{book}
\usepackage[times,inconsolata,hyper]{Rd}
\usepackage{makeidx}
\usepackage[utf8]{inputenc} % @SET ENCODING@
% \usepackage{graphicx} % @USE GRAPHICX@
\makeindex{}
\begin{document}
\chapter*{}
\begin{center}
{\textbf{\huge Package `shinyr'}}
\par\bigskip{\large \today}
\end{center}
\begin{description}
\raggedright{}
\inputencoding{utf8}
\item[Type]\AsIs{Package}
\item[Title]\AsIs{Easy insights through your data}
\item[Version]\AsIs{0.2}
\item[Description]\AsIs{`shinyr` is developed to build dynamic R `shiny` based dashboards to analyze any CSV files.
It provides simple dashboard design to subset the data, perform exploratory data analysis and preliminary machine learning (supervised and unsupervised). It also provides filters based on columns of interest.}
\item[Depends]\AsIs{R (>= 3.1.0),}
\item[Imports]\AsIs{dplyr, shiny, shinydashboard, tm, wordcloud, corrplot,
randomForest, RColorBrewer, DMwR, caret, nnet, plotly}
\item[Maintainer]\AsIs{The package maintainer }\email{itsjay510@gmail.com}\AsIs{}
\item[License]\AsIs{GPL-3}
\item[Encoding]\AsIs{UTF-8}
\item[LazyData]\AsIs{true}
\item[RoxygenNote]\AsIs{6.1.1}
\item[Suggests]\AsIs{testthat}
\item[URL]\AsIs{}\url{https://github.com/rpushker/shinyr}\AsIs{}
\item[NeedsCompilation]\AsIs{no}
\item[Author]\AsIs{Jayachandra N [aut, cre],
Pushker Ravindra [aut]}
\end{description}
\Rdcontents{\R{} topics documented:}
\inputencoding{utf8}
\HeaderA{check\_and\_install}{Check and Install}{check.Rul.and.Rul.install}
%
\begin{Description}\relax
Check whether given packages are installed or not and if not installed install them
\end{Description}
%
\begin{Usage}
\begin{verbatim}
check_and_install(packs)
\end{verbatim}
\end{Usage}
%
\begin{Arguments}
\begin{ldescription}
\item[\code{packs}] Vector of package names
\end{ldescription}
\end{Arguments}
%
\begin{Details}\relax
check\_and\_install
\end{Details}
%
\begin{Value}
data.frame, status of required packages and their installation status
\end{Value}
%
\begin{Examples}
\begin{ExampleCode}
check_and_install(c('dplyr', 'data.table'))
\end{ExampleCode}
\end{Examples}
\inputencoding{utf8}
\HeaderA{confmatrix}{Conf Matrix}{confmatrix}
%
\begin{Description}\relax
Calculates a cross-tabulation of observed and predicted classes with associated statistics.
\end{Description}
%
\begin{Usage}
\begin{verbatim}
confmatrix(actuals, preds)
\end{verbatim}
\end{Usage}
%
\begin{Arguments}
\begin{ldescription}
\item[\code{actuals}] a numeric vector

\item[\code{preds}] a numeric vector
\end{ldescription}
\end{Arguments}
%
\begin{Details}\relax
confmatrix
\end{Details}
%
\begin{Value}
A table same as caret::ConfusionMatrix
\end{Value}
%
\begin{Author}\relax
Jayachandra N
\end{Author}
%
\begin{Examples}
\begin{ExampleCode}
confmatrix(c(1,1,1,0), c(1,1,0,0))
\end{ExampleCode}
\end{Examples}
\inputencoding{utf8}
\HeaderA{dataPartition}{Data Partition}{dataPartition}
%
\begin{Description}\relax
Partition data for training and test
\end{Description}
%
\begin{Usage}
\begin{verbatim}
dataPartition(df, train_data_perc)
\end{verbatim}
\end{Usage}
%
\begin{Arguments}
\begin{ldescription}
\item[\code{df}] data.frame which need to be devided into train and test subset

\item[\code{train\_data\_perc}] numeric value between 1 to 100
\end{ldescription}
\end{Arguments}
%
\begin{Details}\relax
dataPartition
\end{Details}
%
\begin{Value}
list of length 2 which contains Train data and Test data
\end{Value}
%
\begin{Author}\relax
Jayachandra N
\end{Author}
%
\begin{Examples}
\begin{ExampleCode}
dataPartition(iris, 80)
\end{ExampleCode}
\end{Examples}
\inputencoding{utf8}
\HeaderA{detectClass}{Detect Class}{detectClass}
%
\begin{Description}\relax
Detects class of given objects
\end{Description}
%
\begin{Usage}
\begin{verbatim}
detectClass(x)
\end{verbatim}
\end{Usage}
%
\begin{Arguments}
\begin{ldescription}
\item[\code{x}] a vector
\end{ldescription}
\end{Arguments}
%
\begin{Details}\relax
detectClass
\end{Details}
%
\begin{Value}
type of the vector
\end{Value}
%
\begin{Author}\relax
Jayachandra N
\end{Author}
%
\begin{Examples}
\begin{ExampleCode}
detectClass(c(1,2,3))
detectClass(c("a","b"))
detectClass(iris$Species)
\end{ExampleCode}
\end{Examples}
\inputencoding{utf8}
\HeaderA{excludeThese}{Exclude These}{excludeThese}
%
\begin{Description}\relax
Exclude an item from a set of items
\end{Description}
%
\begin{Usage}
\begin{verbatim}
excludeThese(set, items_to_exclude)
\end{verbatim}
\end{Usage}
%
\begin{Arguments}
\begin{ldescription}
\item[\code{set}] vector

\item[\code{items\_to\_exclude}] vector to exclude from the whole set
\end{ldescription}
\end{Arguments}
%
\begin{Details}\relax
excludeThese
\end{Details}
%
\begin{Value}
vector
\end{Value}
%
\begin{Author}\relax
Jayachandra N
\end{Author}
%
\begin{Examples}
\begin{ExampleCode}
excludeThese(1:10, 1)
\end{ExampleCode}
\end{Examples}
\inputencoding{utf8}
\HeaderA{getcharacterCols}{Get Character Cols}{getcharacterCols}
%
\begin{Description}\relax
Get character columns.
\end{Description}
%
\begin{Usage}
\begin{verbatim}
getcharacterCols(dat)
\end{verbatim}
\end{Usage}
%
\begin{Arguments}
\begin{ldescription}
\item[\code{dat}] data frame
\end{ldescription}
\end{Arguments}
%
\begin{Details}\relax
getcharacterCols
\end{Details}
%
\begin{Value}
A Character vector of names of numeric columns of a given data frame
\end{Value}
%
\begin{Author}\relax
Jayachandra N
\end{Author}
%
\begin{Examples}
\begin{ExampleCode}
getcharacterCols(iris)
getcharacterCols(mtcars)
\end{ExampleCode}
\end{Examples}
\inputencoding{utf8}
\HeaderA{getCoefficients}{Get Coefficients}{getCoefficients}
%
\begin{Description}\relax
Get coefficients from the model summary
\end{Description}
%
\begin{Usage}
\begin{verbatim}
getCoefficients(model)
\end{verbatim}
\end{Usage}
%
\begin{Arguments}
\begin{ldescription}
\item[\code{model}] lm model
\end{ldescription}
\end{Arguments}
%
\begin{Details}\relax
getCoefficients
\end{Details}
%
\begin{Value}
data.frame of coeffcients
\end{Value}
%
\begin{Author}\relax
Jayachandra N
\end{Author}
%
\begin{Examples}
\begin{ExampleCode}
 model <- lm(Sepal.Length ~ ., iris) # A linear regression model
 getCoefficients(model)
\end{ExampleCode}
\end{Examples}
\inputencoding{utf8}
\HeaderA{getDataInsight}{get Data Insights}{getDataInsight}
%
\begin{Description}\relax
Get detailed insights about the data like number of rows, columns and some basic statistics such as mean
\end{Description}
%
\begin{Usage}
\begin{verbatim}
getDataInsight(temp)
\end{verbatim}
\end{Usage}
%
\begin{Arguments}
\begin{ldescription}
\item[\code{temp}] data frame
\end{ldescription}
\end{Arguments}
%
\begin{Details}\relax
getDataInsight
\end{Details}
%
\begin{Value}
list of details of data
\end{Value}
%
\begin{Author}\relax
Jayachandra N
\end{Author}
%
\begin{Examples}
\begin{ExampleCode}
getDataInsight(mtcars)
getDataInsight(iris)
\end{ExampleCode}
\end{Examples}
\inputencoding{utf8}
\HeaderA{getFeqTable}{Get Freq Table}{getFeqTable}
%
\begin{Description}\relax
Get frequency table for a given text
\end{Description}
%
\begin{Usage}
\begin{verbatim}
getFeqTable(text)
\end{verbatim}
\end{Usage}
%
\begin{Arguments}
\begin{ldescription}
\item[\code{text}] plain text or a paragraph
\end{ldescription}
\end{Arguments}
%
\begin{Details}\relax
getFeqTable
\end{Details}
%
\begin{Value}
data frame of word and it's frequency.
\end{Value}
%
\begin{Author}\relax
Jayachandra N
\end{Author}
%
\begin{Examples}
\begin{ExampleCode}
getFeqTable("shinyr is Incredible!")
\end{ExampleCode}
\end{Examples}
\inputencoding{utf8}
\HeaderA{getLibraryReport}{Get Library Report}{getLibraryReport}
%
\begin{Description}\relax
Get report on whether the given packages are installed on not
\end{Description}
%
\begin{Usage}
\begin{verbatim}
getLibraryReport(packages)
\end{verbatim}
\end{Usage}
%
\begin{Arguments}
\begin{ldescription}
\item[\code{packages}] Vector of package names
\end{ldescription}
\end{Arguments}
%
\begin{Details}\relax
getLibraryReport
\end{Details}
%
\begin{Value}
data.frame, status of required packages and their installation status
\end{Value}
%
\begin{Author}\relax
Jayachandra N
\end{Author}
%
\begin{Examples}
\begin{ExampleCode}
check_and_install(c('dplyr', 'data.table'))
\end{ExampleCode}
\end{Examples}
\inputencoding{utf8}
\HeaderA{getMostRepeatedValue}{Get Most Repeated Value}{getMostRepeatedValue}
%
\begin{Description}\relax
get most repeated value in a given vector.
\end{Description}
%
\begin{Usage}
\begin{verbatim}
getMostRepeatedValue(vec)
\end{verbatim}
\end{Usage}
%
\begin{Arguments}
\begin{ldescription}
\item[\code{vec}] Vector to calculate most repeated values
\end{ldescription}
\end{Arguments}
%
\begin{Details}\relax
getMostRepeatedValue
\end{Details}
%
\begin{Value}
most repeated values in the given set of values
\end{Value}
%
\begin{Author}\relax
Jayachandra N
\end{Author}
%
\begin{Examples}
\begin{ExampleCode}
getMostRepeatedValue(c(1,2,3,3,3,2))
getMostRepeatedValue(c("R", "R", "Python", "Python", "R"))
\end{ExampleCode}
\end{Examples}
\inputencoding{utf8}
\HeaderA{getnumericCols}{Get Numeric Cols}{getnumericCols}
%
\begin{Description}\relax
Get all columns which are numeric.
\end{Description}
%
\begin{Usage}
\begin{verbatim}
getnumericCols(dat)
\end{verbatim}
\end{Usage}
%
\begin{Arguments}
\begin{ldescription}
\item[\code{dat}] data frame
\end{ldescription}
\end{Arguments}
%
\begin{Details}\relax
getnumericCols
\end{Details}
%
\begin{Value}
Character vector of names of numeric columns of given data frame
\end{Value}
%
\begin{Author}\relax
Jayachandra N
\end{Author}
%
\begin{Examples}
\begin{ExampleCode}
getnumericCols(iris)
getnumericCols(mtcars)
\end{ExampleCode}
\end{Examples}
\inputencoding{utf8}
\HeaderA{getType}{Get Type}{getType}
%
\begin{Description}\relax
getType
\end{Description}
%
\begin{Usage}
\begin{verbatim}
getType(vec)
\end{verbatim}
\end{Usage}
%
\begin{Arguments}
\begin{ldescription}
\item[\code{vec}] A vector of any choice, to detect between numeric or character
\end{ldescription}
\end{Arguments}
%
\begin{Value}
type of the given vector
\end{Value}
%
\begin{Author}\relax
Jayachandra N
\end{Author}
%
\begin{Examples}
\begin{ExampleCode}
getType(iris$Species)
getType(as.factor(c(1,0,1,1,0,NA,1, NULL)))
getType(as.factor(c(1, NULL,0,1,1,0,1,'a')))
getType(c(1,2,3,4, NA))
getType(letters[1:4])
\end{ExampleCode}
\end{Examples}
\inputencoding{utf8}
\HeaderA{getWordCloud}{Get Word Cloud}{getWordCloud}
%
\begin{Description}\relax
Get word cloud for given table of words' frequencies
\end{Description}
%
\begin{Usage}
\begin{verbatim}
getWordCloud(d)
\end{verbatim}
\end{Usage}
%
\begin{Arguments}
\begin{ldescription}
\item[\code{d}] table of word's frequency
\end{ldescription}
\end{Arguments}
%
\begin{Details}\relax
getWordCloud
\end{Details}
%
\begin{Value}
Word cloud plot
\end{Value}
%
\begin{Examples}
\begin{ExampleCode}
x <- getFeqTable("Hello! R is Great")
getWordCloud(x)
\end{ExampleCode}
\end{Examples}
\inputencoding{utf8}
\HeaderA{groupByandSumarize}{Group By And Summarize}{groupByandSumarize}
%
\begin{Description}\relax
Group by columns and summarize given data.
\end{Description}
%
\begin{Usage}
\begin{verbatim}
groupByandSumarize(df, grp_col, summarise_col, FUN = mean)
\end{verbatim}
\end{Usage}
%
\begin{Arguments}
\begin{ldescription}
\item[\code{df}] data frame

\item[\code{grp\_col}] column name to group

\item[\code{summarise\_col}] column name to summarize

\item[\code{FUN}] function to summarize
\end{ldescription}
\end{Arguments}
%
\begin{Details}\relax
groupByandSumarize
\end{Details}
%
\begin{Value}
summarized table
\end{Value}
%
\begin{Author}\relax
Jayachandra N
\end{Author}
%
\begin{Examples}
\begin{ExampleCode}
groupByandSumarize(mtcars, grp_col = c("am"), summarise_col = "hp", FUN = "mean")
\end{ExampleCode}
\end{Examples}
\inputencoding{utf8}
\HeaderA{imputeMyData}{Impute My Data}{imputeMyData}
%
\begin{Description}\relax
Impute for missing values in given column in a given data by given method.
\end{Description}
%
\begin{Usage}
\begin{verbatim}
imputeMyData(df, col, FUN)
\end{verbatim}
\end{Usage}
%
\begin{Arguments}
\begin{ldescription}
\item[\code{df}] data frame to impute

\item[\code{col}] a column name of data frame to impute

\item[\code{FUN}] a function to be used for imputing values one of(mean, median, sum, min, max)
\end{ldescription}
\end{Arguments}
%
\begin{Details}\relax
imputeMyData
\end{Details}
%
\begin{Value}
data frame after imputing the values
\end{Value}
%
\begin{Author}\relax
Jayachandra N
\end{Author}
%
\begin{Examples}
\begin{ExampleCode}
x <- head(iris)
x$Sepal.Length[1] <- NA
imputeMyData(x, "Sepal.Length", "mean")
\end{ExampleCode}
\end{Examples}
\inputencoding{utf8}
\HeaderA{make\_var}{Make Var}{make.Rul.var}
%
\begin{Description}\relax
Make a variable from a given character vector.
\end{Description}
%
\begin{Usage}
\begin{verbatim}
make_var(prefix, var, suffix)
\end{verbatim}
\end{Usage}
%
\begin{Arguments}
\begin{ldescription}
\item[\code{prefix}] prefix character

\item[\code{var}] character to convert

\item[\code{suffix}] suffix character
\end{ldescription}
\end{Arguments}
%
\begin{Details}\relax
make\_var
\end{Details}
%
\begin{Value}
variable
\end{Value}
%
\begin{Author}\relax
Jayachandra N
\end{Author}
%
\begin{Examples}
\begin{ExampleCode}
make_var("", "Jay", "")
make_var("", "Incredible_India", "")
\end{ExampleCode}
\end{Examples}
\inputencoding{utf8}
\HeaderA{missing\_count}{Missing Count}{missing.Rul.count}
%
\begin{Description}\relax
Count the number of missing values in a vector.
\end{Description}
%
\begin{Usage}
\begin{verbatim}
missing_count(x)
\end{verbatim}
\end{Usage}
%
\begin{Arguments}
\begin{ldescription}
\item[\code{x}] vector
\end{ldescription}
\end{Arguments}
%
\begin{Details}\relax
missing\_count
\end{Details}
%
\begin{Value}
Number of missing values in the given set of values
\end{Value}
%
\begin{Author}\relax
Jayachandra N
\end{Author}
%
\begin{Examples}
\begin{ExampleCode}
missing_count(c(1,2,3))
missing_count(c(NA, 1, NA, "NULL", ""))
\end{ExampleCode}
\end{Examples}
\inputencoding{utf8}
\HeaderA{multinomial}{Multinomial}{multinomial}
%
\begin{Description}\relax
Fit Multinomial Log-linear Models.
\end{Description}
%
\begin{Usage}
\begin{verbatim}
multinomial(eqn, df)
\end{verbatim}
\end{Usage}
%
\begin{Arguments}
\begin{ldescription}
\item[\code{eqn}] formula to build model

\item[\code{df}] data frame
\end{ldescription}
\end{Arguments}
%
\begin{Details}\relax
multinomial
\end{Details}
%
\begin{Value}
model
\end{Value}
%
\begin{Author}\relax
Jayachandra N
\end{Author}
%
\begin{Examples}
\begin{ExampleCode}
multinomial( Species ~ .,  iris)
\end{ExampleCode}
\end{Examples}
\inputencoding{utf8}
\HeaderA{plotCor}{Plot Cor}{plotCor}
%
\begin{Description}\relax
Plot correlation plot
\end{Description}
%
\begin{Usage}
\begin{verbatim}
plotCor(cor_dat, my_method)
\end{verbatim}
\end{Usage}
%
\begin{Arguments}
\begin{ldescription}
\item[\code{cor\_dat}] Corelation matrix

\item[\code{my\_method}] method to plot, for example: circle
\end{ldescription}
\end{Arguments}
%
\begin{Details}\relax
plotCor
\end{Details}
%
\begin{Value}
Corelation plot
\end{Value}
%
\begin{Author}\relax
Jayachandra N
examples
cor\_dat <- cor(mtcars)
plotCor(cor\_dat, "circle")
\end{Author}
\inputencoding{utf8}
\HeaderA{randomForestModel}{Random Forest Model}{randomForestModel}
%
\begin{Description}\relax
Build Random Forest Model.
\end{Description}
%
\begin{Usage}
\begin{verbatim}
randomForestModel(eqn, df)
\end{verbatim}
\end{Usage}
%
\begin{Arguments}
\begin{ldescription}
\item[\code{eqn}] formula

\item[\code{df}] data.frame
\end{ldescription}
\end{Arguments}
%
\begin{Details}\relax
randoMForestModel
\end{Details}
%
\begin{Value}
rf model
\end{Value}
%
\begin{Author}\relax
Jayachandra N
\end{Author}
%
\begin{Examples}
\begin{ExampleCode}
randomForestModel( Species ~ .,  iris)
\end{ExampleCode}
\end{Examples}
\inputencoding{utf8}
\HeaderA{regressionModelMetrics}{Regression Model Metrics}{regressionModelMetrics}
%
\begin{Description}\relax
Generate regression model metrics such as R-squared and MAPE.
\end{Description}
%
\begin{Usage}
\begin{verbatim}
regressionModelMetrics(actuals, predictions, model)
\end{verbatim}
\end{Usage}
%
\begin{Arguments}
\begin{ldescription}
\item[\code{actuals}] numeric vector of actual values

\item[\code{predictions}] numeric vector of predictions

\item[\code{model}] lm model object
\end{ldescription}
\end{Arguments}
%
\begin{Details}\relax
regressionModelMetrics
\end{Details}
%
\begin{Value}
list
\end{Value}
%
\begin{Author}\relax
Jayachandra N
\end{Author}
%
\begin{Examples}
\begin{ExampleCode}
## Not run: 
mod <- lm(formula = wt ~ ., data = mtcars)
predictions <- predict(mod, mtcars[,-6])
actials <- mtcars[,6]
regressionModelMetrics(actuals = actials,
predictions = predictions, model = mod)

## End(Not run)
\end{ExampleCode}
\end{Examples}
\inputencoding{utf8}
\HeaderA{shineMe}{shineMe}{shineMe}
%
\begin{Description}\relax
An R shiny app for shinyr UI.
\end{Description}
%
\begin{Usage}
\begin{verbatim}
shineMe()
\end{verbatim}
\end{Usage}
%
\begin{Details}\relax
shineMe
\end{Details}
%
\begin{Value}
shiny UI page
\end{Value}
%
\begin{Author}\relax
Jayachandra N
\end{Author}
%
\begin{Examples}
\begin{ExampleCode}
## Not run: 
shineMe()

## End(Not run)
\end{ExampleCode}
\end{Examples}
\inputencoding{utf8}
\HeaderA{splitAndGet}{Split And Get}{splitAndGet}
%
\begin{Description}\relax
Split a string by space and get
\end{Description}
%
\begin{Usage}
\begin{verbatim}
splitAndGet(x)
\end{verbatim}
\end{Usage}
%
\begin{Arguments}
\begin{ldescription}
\item[\code{x}] string to split into words
\end{ldescription}
\end{Arguments}
%
\begin{Details}\relax
splitAndGet
\end{Details}
%
\begin{Value}
List of worrds
\end{Value}
%
\begin{Author}\relax
Jayachandra N
\end{Author}
%
\begin{Examples}
\begin{ExampleCode}
splitAndGet("R programming is awesome!")
\end{ExampleCode}
\end{Examples}
\inputencoding{utf8}
\HeaderA{valid\_sets}{Valid Sets}{valid.Rul.sets}
%
\begin{Description}\relax
Get a list of all datasets available as data.frame in R
\end{Description}
%
\begin{Usage}
\begin{verbatim}
valid_sets(package = NULL, cols = NULL)
\end{verbatim}
\end{Usage}
%
\begin{Arguments}
\begin{ldescription}
\item[\code{package}] package name to fetch inbuilt data sets example:  "datasets"

\item[\code{cols}] numeric to specify condition on how many columns should data frame have
\end{ldescription}
\end{Arguments}
%
\begin{Details}\relax
valid\_sets
\end{Details}
%
\begin{Value}
data frame all available datasets of class data frame
\end{Value}
%
\begin{Author}\relax
Pushker Ravindra

Jayachandra N
\end{Author}
%
\begin{Examples}
\begin{ExampleCode}
valid_sets()
\end{ExampleCode}
\end{Examples}
\printindex{}
\end{document}
